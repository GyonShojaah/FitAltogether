\documentclass[iop,numberedappendix,apj,]{emulateapj}

\usepackage{epsfig}
%\usepackage{amsmath}
\usepackage{amsmath,amsthm,amssymb,cases}
\usepackage{rotating}
\usepackage{natbib}
\usepackage{enumerate}
\usepackage{multirow}
\usepackage{array}
\usepackage{appendix}
\usepackage{comment}
\usepackage{color,xcolor}
\usepackage{url}
\usepackage{here}
\usepackage{hyperref}
\hypersetup{colorlinks,linkcolor={blue!50!black},citecolor={blue!50!black},urlcolor={blue!50!black}}
\allowdisplaybreaks[1]
\bibliographystyle{apj}
\renewcommand{\bibname}{References}

\def\plotonesc#1{\centering \leavevmode
\includegraphics[clip=, width=1.70\columnwidth]{#1}}
\def\plotoneh#1{\centering \leavevmode
\includegraphics[clip=, width=.95\columnwidth]{#1}}
\def\plotone#1{\centering \leavevmode
\includegraphics[clip=, width=.85\columnwidth]{#1}}
\def\plotoneShrinkSmall#1{\centering \leavevmode
\includegraphics[clip=, width=.49\columnwidth]{#1}}
\def\plotoneShrinkMed#1{\centering \leavevmode
\includegraphics[clip=, width=.55\columnwidth]{#1}}
\def\plotoneShrinkBig#1{\centering \leavevmode
\includegraphics[clip=, width=.65\columnwidth]{#1}}
\def\plottwo#1#2{\centering \leavevmode
\includegraphics[width=.45\columnwidth]{#1} \hfil
\includegraphics[width=.45\columnwidth]{#2}}
\def\plottwob#1#2{\centering \leavevmode
\includegraphics[width=.49\columnwidth]{#1} \hfil
\includegraphics[width=.49\columnwidth]{#2}}
\def\plottwor#1#2{\centering \leavevmode
\includegraphics[width=.55\columnwidth,angle=90]{#1} \hfil
\includegraphics[width=.55\columnwidth,angle=90]{#2}}
\def\plotthree#1#2#3{\centering \leavevmode
\includegraphics[width=.3\columnwidth]{#1} \hfil
\includegraphics[width=.3\columnwidth]{#2} \hfil
\includegraphics[width=.3\columnwidth]{#3}}

\def\gsim{\;\rlap{\lower 2.5pt
 \hbox{$\sim$}}\raise 1.5pt\hbox{$>$}\;}
\def\lsim{\;\rlap{\lower 2.5pt
   \hbox{$\sim$}}\raise 1.5pt\hbox{$<$}\;}
%\def\fast{\;$f^{\ast }$\;}
\def\fast{\tilde f}

% set formatting properties
\setlength{\textwidth}{6.5in}
\setlength{\textheight}{8.8in}
\setlength{\hoffset}{0.0in}
\setlength{\voffset}{-0.4in}
\parindent 0.2in
\parskip 0.1in

\def\memoYF#1{\color{red}[YF: {\bf #1}]\color{black}}
\def\memoJLY#1{\color{green}[JLY: {\bf #1}]\color{black}}
\def\memoNBC#1{\color{blue}[NBC: {\bf #1}]\color{black}}



%%% http://www.oceanwave.jp/index.php?float%B4%C4%B6%AD(figure%2Ftable)%A4%CE%BD%D0%CE%CF%B0%CC%C3%D6%A4%F2%A5%B3%A5%F3%A5%C8%A5%ED%A1%BC%A5%EB

%%%%%%%%%%%%%%%%%%%%%%%%%%%%%%%%%%%%%%%%%%%%%%%%%
% THE DOCUMENT BEGINS HERE                      %
%%%%%%%%%%%%%%%%%%%%%%%%%%%%%%%%%%%%%%%%%%%%%%%%%

%\slugcomment{Submitted to ApJ, XX September 2015}

\begin{document}

\title{Rotational Spectral Unmixing of Exoplanets:\\Degeneracies between Surface Colors and Geography}


\author{
%
Yuka Fujii\altaffilmark{1,2} 
%
Jacob Lustig-Yaeger\altaffilmark{3,4,5} 
%
Nicolas B. Cowan\altaffilmark{6,7} 
%
}

\affil{$^1$NASA Goddard Institute for Space Studies, 
  New York, NY 10025, USA}
      
\affil{$^2$Earth-Life Science Institute, Tokyo Institute of Technology, 
  Tokyo, 152-8550, JAPAN}
  
\affil{$^3$Astronomy Department, University of Washington, Box 951580, Seattle, WA 98195, USA}

\affil{$^4$Astrobiology Program, University of Washington, 3910 15th Ave. NE, Box 351580, Seattle, WA 98195, USA}

\affil{$^5$NASA Astrobiology Institute -- Virtual Planetary Laboratory Lead Team, USA}

\affil{$^6$Department of Earth and Planetary Sciences, McGill University, Montreal, Quebec
Canada H3A 0E8}

\affil{$^7$Department of Physics, McGill University, Montreal, Quebec
Canada H3A 0E8}



\vspace{0.5\baselineskip}

\email{
yuka.fujii.ebihara@gmail.com
}

\begin{abstract}

\end{abstract}

\keywords{planets and satellites: surfaces --- planets and satellites: terrestrial planets}
  
%]%%% End front material



%%%%%%%%%%%%%%%%%%%%%%%%%%%%%%%%%%%%%%%%%%%%%%%%%%%%%%%%%%%%%%%%%%%
\section{Introduction}
\label{sec:intro}
%%%%%%%%%%%%%%%%%%%%%%%%%%%%%%%%%%%%%%%%%%%%%%%%%%%%%%%%%%%%%%%%%%%

Direct imaging of exoplanets is expected to play a vital role in characterizing Earth analogs in habitable zones and beyond. 
Substantial work has gone into prediciting detectable features in disk-integrated spectra of the Earth and other planets, as they are observed from an astronomical distance. 
Atmospheric molecules are identifiable through spectral absorption features \citep[e.g.,][]{DesMarais2002} but surface reflectance spectra also affect the spectra, which could be measured through low-resolution spectra, or multi-band photometry. 
However, interpreting disk-integrated colors is not trivial. 
This is particularly true for Earth-like planets that harbor diverse atmospheric and surface characteristics including liquid water, partial cloud cover, continents, and possibly vegetation. 

A key here is to leverage the time variation of the spectrum \citep{Ford2001}: the regions that contributes to the scattered light change due to the planetary rotation and  orbital motion, so the time variability can in principle be translated to the heterogeneity of the surface environment.  
\citet{Cowan2009, Cowan2011} performed Principal Component Analysis (PCA) on the observed multi-band photometry of Earth. They found that the number of surface types can be inferred from the number of dominant principle components: (\# of surface types) $\ge $ (\# of principal components) $+ 1$. %) and that the variation pattern is indicative of the surface properties. 
\citet{Fujii2010, Fujii2011} decomposed multi-band photometric data of the Earth assuming the template reflectance spectra of the known major surface types, and they found that the relative abundance and the longitudinal variation of these surface types are approximately recovered. 
Moreover, by coupling the time variation due to rotation with phase variation due to orbital revolution, a 2-dimensional map of the surface may be retrieved \citep{Kawahara2010, Kawahara2011, Fujii2012}. 

\citet{Cowan2013} took another approach to the same inverse problem. 
Their strategy was to try to estimate the reflectance spectra of surfaces and their distribution across the globe simultaneously, by making all of them fitting parameters. 
The result appeared successful in so far as the obtained reflectance spectra roughly match the average spectra of clouds, ocean, and continents. 
However, the longitudinal map of these components did not match the actual geographyical distribution of surfaces. 

We were motivated by this unsatisfactory result to revisit their analysis. 
Specifically, the updates are (1) Revisiting the formulation and point out the inherit degeneracy between surface spectra and geography, and (2) Introducing regularization terms to enforce smooth longitudinal maps. 
%We limit our discussions to the analysis of noise-free data. Clearly, noise in realistic observations of exoplanets are expected to be substantially larger, and the effect of such observations will be discussed elsewhere. 
The organization of this paper is as follows. 
Section \ref{s:frame} revisits the problem, ....

%\newpage

%%%%%%%%%%%%%%%%%%%%%%%%%%%%%%%%%%%
\begin{figure}[t]
    \begin{center}
	\includegraphics[width=\hsize]{mockdata.pdf}
%    \includegraphics[width=\hsize]{mockdata_3types_albd.pdf}
%	\includegraphics[width=\hsize]{mockdata_3types_t360_lc.pdf}
    \end{center}
    \caption{Our mock data based on IGBP classification map of the Earth. Top panel: distribution of 3 surface types (ocean: blue, soil: red, vegetation: green). Middle panel: assumed albedo spectra with matching colors. Bottom 4 panels: rotational light curves in 4 photometric band with varying phase angle, $\alpha = 135^{\circ }$ (purple, dot-dashed), $90^{\circ }$ (olive, solid), and $45^{\circ }$ (gold, dashed). }
\label{fig:mockdata}
\end{figure}
%%%%%%%%%%%%%%%%%%%%%%%%%%%%%%%%%%%


%%%%%%%%%%%%%%%%%%%%%%%%%%%%%%%%%%%%%%%%%%%%%%%%%%%%%%%%%%%%%%%%%%%
\section{Preparing Mock Datasets}
\label{s:mockdata}
%%%%%%%%%%%%%%%%%%%%%%%%%%%%%%%%%%%%%%%%%%%%%%%%%%%%%%%%%%%%%%%%%%%

%The aim of this paper is to present how to constrain the albedo spectra of representative surface types from multi-band light curves. 
In order to facilitate the discussions in the following sections, we shall introduce a mock dataset to be used for demonstrations in this section. 

We consider diurnal light curves of a toy model of the atmosphere-less Earth, in 4 photometric bands. 
We use a simplified surface map as shown in the upper left panel of Figure \ref{fig:mockdata}. 
This map is based on the land classification by the International Geosphere-Biosphere Programme (IGBP)\footnote{\url{https://climatedataguide.ucar.edu/climate-data/ceres-igbp-land-classification}}. 
Although the original classification assumes 16 land surface types plus ocean, in this paper we adopt 3 surface types for simplicity regarding ``Open Shrubs'', ``Urban'', ``Snow/Ice'', and ``Barren/Desert'' as ``soil'' (red in the upper left panel of Figure \ref{fig:mockdata}) and other land surface types as ``vegetation'' (green), while keeping the ``ocean'' regions. 


The assumed albedo spectra of these surface types in 4 photometric bands are shown in the upper right panel of Figure \ref{fig:mockdata}. 
These 4 photometric bands actually correspond to 0.4-0.5 $\mu $m, 0.5-0.6 $\mu $m, 0.6-0.7 $\mu $m, and 0.7-0.8 $\mu $m, respectively, but they will be simply called by the band indices unless otherwise noted. 
%\memoJLY{I think it would be helpful to make the "band \#"-axis wavelength in microns, and then perhaps refer to the band \# in the text or in an upper x-axis. }
The albedo spectrum for ocean is based on \citet{Mclinden1997}, 
and the data for soil and vegetation are taken from ASTER spectral library\footnote{\url{https://speclib.jpl.nasa.gov/}. 
Specifically, we adopt  ``Brown to dark brown sand'' for ``soil'', and ``Grass'' for the ``vegetation''}. 
The scattering phase function by the surface is assumed to obey the Lambert law, i.e., the radiance is (incident flux)$\times $(albedo)/$\pi$, independent of the direction of the scattering. 
Note that in reality they are not Lambertian scatterers; among others,  scattering by the ocean is very anisotropic (Section \ref{ss:deviateLambert}).  

% and the albedo spectra of ocean overlaid with atmosphere  
%\memoJLY{A phase of 135 deg is right in the range that Ty (Robinson et al. 2010; 2011) shows will deviate most from a Lambertian scatterer due the ocean glint...}

{\color{red} 
The diurnal light curves are synthesized given the relative positions of the star, planet, and observer. 
% We consider light curves of 1 rotation at a fixed orbital phases. 
For the sake of simplicity, we consider a planet with zero obliquity in an edge-on orbit, and change the phase angle (the planet-centric angle between the star and the observer) denoted by $\alpha $. 
We also assume that spin period is significantly shorter than the orbital period and the orbital phase does not change in 1 spin rotation. 
However, the following discussion does not depend on these assumptions. 
We consider noise-free data. Clearly, noise in realistic observations of exoplanets are expected to be substantially large, and the effect of such observations will be discussed elsewhere.}
\color{black}

The bottom panels of Figure \ref{fig:mockdata} display the examples of diurnal light curves in 4 photometric bands, represented in terms of {\it apparent albedo}. 
Apparent albedo is the albedo of the planet weighted by illumination and visibility. 
More specifically, it is obtained by normalizing the planetary intensity by that of a loss-less Lambert sphere with the same radius and at the same phase \citep{Qiu2003, Seager2010}. 
In this paper we will use apparent albedo unless otherwise noted. 
The apparent albedo is straightforwardly obtained from the observed planetary intensity, if and only if the planetary radius and the observational geometry are completely known: latitude/longitude of the sub-stellar/sub-observer points, and the distance between the star and the planet. 

The problem throughout this paper is, from these kinds of multi-band diurnal light curves, how and how well we can retrieve the albedo spectra of different surface types and the longitudinal distribution of these surface types. 


%%%%%%%%%%%%%%%%%%%%%%%%%%%%%%%%%%%%%%%%%%%%%%%%%%%%%%%%%%%%%%%%%%%
\section{Inverse problem}
\label{s:frame}
%%%%%%%%%%%%%%%%%%%%%%%%%%%%%%%%%%%%%%%%%%%%%%%%%%%%%%%%%%%%%%%%%%%

In this section, we discuss the general framework to analyze the diurnal light curves, and present some demonstrations using the mock data created in the previous section. 
Sections \ref{ss:model} and \ref{ss:PCplane} are essentially the recapitulation of previous papers, in particular \citet{Cowan2013} \citep[but see also][]{Cowan2009,Cowan2011,Fujii2010,Fujii2011}.  
We choose to include these discussions, however, as a baseline to establish the later arguments. 

%%%%%%%%%%%%%%%%%%%%%%%%%%%%%%%%%%%%%%%%%%%%%%%%%%%%%%%%%%%%%%
\subsection{Algebraic Formulation}
\label{ss:model}
%%%%%%%%%%%%%%%%%%%%%%%%%%%%%%%%%%%%%%%%%%%%%%%%%%%%%%%%%%%%%%


%%%%%%%%%%%%%%%%%%%%%%%%%%%%%%%%%%%
\begin{table}[b]
\caption{Indexes}
\begin{center}
\begin{tabular}{lcc} \hline \hline
Name & Symbol & Maximum \\ \hline
Observation Time & $i$ & I \\
Band & $j$ & J  \\
Surface Type & $k$ & K  \\
Longitudinal Slice  & $l$ & L \\ \hline
\end{tabular}
\end{center}
\label{tab:index}
\end{table}%
%%%%%%%%%%%%%%%%%%%%%%%%%%%%%%%%%%%


On the assumption that the planetary surface is everywhere Lambertian scatterer, and that it is composed of a certain number $K$ of spectrally distinct surface types, the disk-integrated scattered light is a weighted summation of the reflectance spectra of different surface types. 
Using the local surface albedo $s_{\vec \Omega }$, the zenith angle of the insolation, $\theta _0$, and the zenith angle of the observer $\theta _1$ (both defined at each surface point),
the apparent albedo of the planet, $d_{ij}$ (``$d$'' for data) is written as follows \citep[see][]{Fujii2010}: 
%%%
\begin{eqnarray}
d_{i} (\lambda_j) &=& \displaystyle \frac{ \int_{{\rm IV}_i} s_{\vec \Omega }(\lambda_j) \cdot \cos \theta_0 ({\vec \Omega}) \cdot \cos \theta_1 ({\vec \Omega}) \cdot d\vec \Omega }{ \int_{{\rm IV}_i}  \cos \theta_0 ({\vec \Omega}) \cdot \cos \theta_1 ({\vec \Omega}) \cdot d\vec \Omega } \\
&=& \sum _{k} s_k (\lambda_j) \; \displaystyle \frac{ \int_{{\rm IV}_{i}} f_k (\vec \Omega ) \cos \theta_0 ({\vec \Omega}) \cdot \cos \theta_1 ({\vec \Omega}) \cdot d\vec \Omega }{ \int_{{\rm IV}_i}  \cos \theta_0 ({\vec \Omega}) \cdot \cos \theta_1 ({\vec \Omega}) \cdot d\vec \Omega } \notag \\
&=& \sum _{i,k} \fast_{ik} \, s_{kj} \label{eq:tilde_d_f_ast_s}
\end{eqnarray}
%%%
where the surface spectra are:
%%%
\begin{equation}
s _{kj} \equiv  s_k (\lambda _j)
\end{equation}
%%%
and the apparent covering fractions are:
%%%
\begin{equation}
\tilde f_{ik} \equiv  \frac{ \int_{{\rm IV}_{i}} f_k (\vec \Omega ) \cos \theta_0 ({\vec \Omega}) \cdot \cos \theta_1 ({\vec \Omega}) \cdot d\vec \Omega }{ \int_{{\rm IV}_i}  \cos \theta_0 ({\vec \Omega}) \cdot \cos \theta_1 ({\vec \Omega}) \cdot d\vec \Omega }
\end{equation}
%%%
and $i$, $j$, and $k$ are indices for the observation epochs, bands, and the surface types, respectively, ${\rm IV }_i$ denotes the illuminated and visible area over the planetary surface at $i$-th observation, and $f_k (\vec \Omega )$ is the area fraction of $k$-th surface type in $d\vec \Omega$. 
In the last expression, $\fast_{ik}$ represents the apparent covering fraction of the $k$-th surface type at $i$-th observational epoch, and 
$s_{kj}$ is the reflectance spectra of $k$-th surface type at $j$-th band. 
The maximum number of $i$, $j$ and $k$ will be denoted by $I$, $J$, and $K$, below, as summarized in Table \ref{tab:index}. 

By definition, the area fractions, $\fast $, should not be negative and they should sum up to unity, and reflectance spectra, $s$, should be between 0 and 1. 
Therefore, the constraints on geography ($\fast_{ik}$) and colors ($s_{kj}$) of the surfaces are:
%%%
\begin{subnumcases}
{}
0 \leq \fast_{lk} \;\;\; & \mbox{for any $l$, $k$} \label{eq:tilde_f_range} \\
\sum_k \fast_{lk} = 1 & \mbox{for any $l$} \label{eq:tilde_f_sum} \\
0 \leq s_{kj} \leq 1 \;\;\; & \mbox{for any $k$, $j$} \label{eq:tilde_s_range} 
\end{subnumcases}
%%%


%%%
%\begin{eqnarray}
%\begin{cases}
%\;\; 0 \leq s_{kj} \leq 1 \;\;\; & \mbox{for any $k$, $j$} \\
%\;\; 0 \leq \fast_{lk} \leq 1 \;\;\; & \mbox{for any $l$, $k$} \label{eq:cond_f_ast}\\
%\;\; \sum_k \fast_{lk} = 1 & \mbox{for any $l$} 
%\end{cases}
%\end{eqnarray}
%%%

%\subsection{Estimating the Composition\\of Longitudinal Slices}

The time variability of the apparent covering fraction $\fast $ due to the planet's rotation is related to the surface inhomogeneity along the equator. Approximately, $\fast _{ik}$ may be written as the weighted summation of the area fraction of $k$-th surface type in each of (the finite number $L$ of) longitudinal slices, i.e.,
%%%
\begin{eqnarray}
\fast _{ik} &=& \frac{ \int f_k (\vec \Omega ) \cos \theta_0 ({\vec \Omega}) \cdot \cos \theta_1 ({\vec \Omega}) \cdot d\vec \Omega }{ \int_{{\rm IV}_i}  \cos \theta_0 ({\vec \Omega}) \cdot \cos \theta_1 ({\vec \Omega}) \cdot d\vec \Omega }  \\
&=& \sum_l \frac{ \int f_k (\vec \Omega ) W_i (\vec \Omega  ) \cdot d\vec \Omega }{ \int  W_i (\vec \Omega ) \cdot d\vec \Omega } \\
%&\approx & \sum_l f_{lk} \frac{ \int W_i (\vec \Omega ) \cdot d\vec \Omega }{ \int  W_i (\vec \Omega ) \cdot d\vec \Omega } \label{eq:discretize}\\
&\approx & \sum_l  W_{il} f_{lk} \label{eq:Wf}, \\
W_{il} &\equiv & \frac{ \int  W_i (\vec \Omega  ) \cdot d\vec \Omega_l }{ \int W_i (\vec \Omega )  \cdot d\vec \Omega }
\end{eqnarray}
%%%

where $l$ is the index for longitudinal slices, $f_{lk}$ is the area fraction of the $k$-th surface type in the $l$-th longitudinal slice. 
Strictly speaking, the approximation is valid only when $f_k(\vec \Omega)$ does not change or changes little across the $l$-th slice for all $k$ and $l$. 

In the last equation, $W_{il}$ is the weight function for $i$-th epoch and $l$-th longitudinal slice which depends only on the observational geometry. 
\citet{Cowan2013} refer to this as the convolution kernel, $K$. 
As a result,
%%%
\begin{equation}
d_{ij} \approx \sum _{l,k} W_{il} \, f_{lk} \, s_{kj} \label{eq:d_f_s}
\end{equation}
%%%
where $f_{lk}$ is the average area fraction of $k$-th surface type at $l$-th longitudinal slice, Again, the area fractions at longitudinal slices, $f_{lk}$, cannot be negative and should sum up to unity. Thus, a set of condition similar to Equations (\ref{eq:tilde_s_range})-(\ref{eq:tilde_f_sum}) are imposed:
%%%
\begin{subnumcases}
{}
0 \leq f_{lk} \;\;\; & \mbox{for any $l$, $k$} \label{eq:f_range} \\
\sum_k f_{lk} = 1 & \mbox{for any $l$} \label{eq:f_sum} \\
0 \leq s_{kj} \leq 1 \;\;\; & \mbox{for any $k$, $j$} \label{eq:s_range}
\end{subnumcases}
%%%
Now, the relevant problem is, given $d$, estimate $\{f, s\}$ subject to the constraints of Equations (\ref{eq:s_range})-(\ref{eq:f_sum})---this is where \citet{Cowan2013} stood. 


%\citet{Cowan2013} proposed the specific procedure to find the solution for $\{f, s\}$. 
%Namely, 
%(1) perform PCA on the data to reduce the dimensionality as needed, 
%(2) perform simplex shrink-wrapping analysis to find the first guess for $s$, 
%and 
%(3) find the maximum posterior values for $\{f, s\}$ using MCMC algorithm  and setting the first guess for $s$ as the initial condition. 
%(The initial value for $f$ is fixed at 1/(\# of surface types).)

%\newpage


%%%%%%%%%%%%%%%%%%%%%%%%%%%%%%%%%%%
\begin{figure}[b!]
    \begin{center}
\includegraphics[width=\hsize]{schematics.pdf}
    \end{center}
    \caption{Schematic figure to illustrate the relation between surface spectra ($\{{\bf s}_k\} $; filled points) and the observed data ($\{{\bf d}_i\} $; crosses). \memoNBC{Explain ``observed data"" in caption instead of as legend. }}
\label{fig:schematic}
\end{figure}
%%%%%%%%%%%%%%%%%%%%%%%%%%%%%%%%%%%


%%%%%%%%%%%%%%%%%%%%%%%%%%%%%%%%%%%%%%%%%%%%%%%%%%%%%%%%%%%%%%
\subsection{Graphical Conception on \\Principal Component Plane}
\label{ss:PCplane}
%%%%%%%%%%%%%%%%%%%%%%%%%%%%%%%%%%%%%%%%%%%%%%%%%%%%%%%%%%%%%%

%A question is whether the above formulation leads us to a unique solution of either $\{ {\bf \fast},\,{\bf s} \}$ or $\{ {\bf f},\,{\bf s}\}$, given the data matrix, ${\bf d}$. 

Equation (\ref{eq:tilde_d_f_ast_s}) coupled with the conditions (\ref{eq:tilde_f_range}) and (\ref{eq:tilde_f_sum}) indicates the geometrical relationship among ${\bf d}$, ${\bf \fast }$, and ${\bf s}$ in the $J$-th dimensional space, where $\{{\bf d}_i\}$ correspond to the points that are located on the plane defined by $K$ ($<J$) number of surface, $\{{\bf s}_k\} $, and are enclosed by $\{{\bf s}_k \}$. 
%\memoYF{How is it called in one word in mathematics?}
Figure \ref{fig:schematic} graphically shows this relations in the case of $J=3$ and $K=3$. 
Note that the dimension of this plane is $K-1$, so in general it is a hyper-plane.  
%We show this graphically in Figure \ref{fig:trajectory}, using the mock datasets shown in Section \ref{s:mockdata}. 

This plane can be identified through Principal Component Analysis (PCA) \citep{Cowan2009,Cowan2011}, as PCA extracts the major, mutually-orthogonal axes along which the scatter among the data points are significant. 
Consequently, the number of major principal components (PCs) are equal to or less than $K-1$ \citep{Cowan2011}. 
{\color{red} Note that non-Lambertian reflectance, partially transparent cloud cover, or observational noise would in practice ensure that the PC plane has some thickness. }
From the mock light curves prepared in Section \ref{s:mockdata}, we extract two dominant PCs through PCA with other components having virtually zero contributions, consistent with the 3 input surface types.  
The PCs extracted from the light curves at $\alpha = 90^{\circ }$ (olive lines in Figure \ref{fig:mockdata}) are presented in the upper panel of Figure \ref{fig:trajectory}. 
For later use, we denote these PCs by $V_{nj}$ where $n=1$ or $2$  corresponding to the 1st and 2nd PC. 

Using $V_{nj}$, the light curves at $\alpha = 90^{\circ }$ ($d_{ij}$) is projected onto the PC plane by
%%%
\begin{equation}
d_{ij} = \sum_n U_{in} V_{nj} + \bar d_j
\end{equation}
%%%
where $U_{in}$ is the trajectory on the PC plane and $\bar d_j$ is the time average of the colors as the offset. 
The trajectory of the light curves at $\alpha = 90^{\circ }$ onto this PC plane is shown by the olive line in Figure \ref{fig:trajectory}. 
The set of PCs ($V_{nj}$) and the offset ($\bar d_j$) uniquely specifies the PC plane. 

We may plot the trajectories of other light curves at different phase angles on the same plane because the PC plane should be the same plane (under the noiseless condition) because we use the same 3 surface types and assume Lambertian surfaces. 
(Note that PCA on individual light curves does not necessarily find the same PCs, because it depends on how the data points are scattered on the PC plane.)
Thus, the trajectories of the mock light curves at different phase angles are also shown in the lower panel of Figure \ref{fig:trajectory}. 
The color excursions are greater at a larger phase angle, i.e., at crescent phase, and shrinks as the phase angle decreases because a larger surface area is averaged together.  
The points in the figure indicate the input albedo spectra of the three surface types. 
As described above, the trajectories are always inside the triangle defined by these points. 
%Because the apparent albedo at any time of the observations is a linear combination of input surface colors (Equation \ref{eq:tilde_d_f_ast_s}) under the conditions (\ref{eq:tilde_f_range}) and (\ref{eq:tilde_f_sum}), the trajectories should always be inside the triangle defined by these points.  



 
%%%%%%%%%%%%%%%%%%%%%%%%%%%%%%%%%%%
%\begin{figure}[tbh!]
%    \begin{center}
%\includegraphics[width=\hsize]{PCA_V_jn.pdf}
%    \end{center}
%    \caption{Principal components of the light curves shown in the bottom of Figure \ref{fig:mockdata}. }
%\label{fig:PCs}
%\end{figure}
%%%%%%%%%%%%%%%%%%%%%%%%%%%%%%%%%%%

%%%%%%%%%%%%%%%%%%%%%%%%%%%%%%%%%%%
\begin{figure}[t]
    \begin{center}
\includegraphics[width=\hsize]{mockdata_PCplane.pdf}
    \end{center}
    \caption{Upper panel: principal components (PCs) of the light curves at $\alpha = 90^{\circ }$ shown in olive lines in Figure \ref{fig:mockdata}. Lower panel: trajectories of the 4-band light curves on the PC plane set by 2 PCs shown in the upper panel, in the case of $\alpha = 135^{\circ }$ (indigo thick line), $\alpha = 90^{\circ }$ (olive line), and $\alpha = 45^{\circ }$ (gold thin line). Points indicate the input albedo spectra of ocean (blue circle), soil (red square), and vegetation (green triangle) on the PC plane. Points in the gray region violate the condition (\ref{eq:tilde_s_range}); specifically, the left, right, and bottom boundaries are set by $s_{k,4} > 0$, $s_{k,1} > 0$, and $s_{k,3}> 0$, respectively. 
Dotted lines are random triangle that could be solutions (see text). }
    \label{fig:trajectory}
\end{figure}
%%%%%%%%%%%%%%%%%%%%%%%%%%%%%%%%%%%

%%%%%%%%%%%%%%%%%%%%%%%%%%%%%%%%%%%%%%%%%%%%%%%%%%%%%%%%%%%%%%
\subsection{Degeneracy}
\label{ss:degeneracy}
%%%%%%%%%%%%%%%%%%%%%%%%%%%%%%%%%%%%%%%%%%%%%%%%%%%%%%%%%%%%%%

On the other hand, when it comes to the stage where we must estimate the surface spectra given the trajectory/-ies, {\it any} set of $\{ {\bf s}_k \}$ that enclose the data points $\{{\bf d}_i\}$ in the hyperplane can be a solution of Equation (\ref{eq:tilde_d_f_ast_s}) subject to the conditions (\ref{eq:tilde_f_range}) and (\ref{eq:tilde_f_sum}). 
Note that the associated matrix, $\fast _{ik}$, can always be found. 

Additional constraints come from the condition (\ref{eq:tilde_s_range}). 
In Figure \ref{fig:trajectory} any points in the shadowed region are rejected based on this condition: specifically, the left, right, and bottom boundaries are set by $s_{k,4}> 0$, $s_{k,1}> 0$, and $s_{k,3}> 0$. 
While in this particular example the permitted region happens to be a triangle, shape can differ depending on the relative configuration of the boundaries. 
Since each of 4 bands has lower and upper bounds on albedo, $0<s<1$, the physically allowed region of color space is a 4-dimensional hypercube, also known as tesseract. Depending on the orientation of the PC plane with this tesseract, the allowed region may have a variety of geometries. 
In general, the allowed region is a $K-1$ dimensional slice through a $J$-dimensional hypercube. 

%Although this is not sufficient to result in a unique solution, it does impact the precision of rotational spectral unmixing as discussed below. 

In Figure \ref{fig:trajectory} we show two random triangles that enclose the trajectory; these triangles satisfy the conditions (\ref{eq:tilde_f_range}) and (\ref{eq:tilde_f_sum}), as do many others. 
For example, with a large triangle one can have spectrally interesting surface spectra and boring geography (small longitudinal variation in area fractions) {\it or} with a small triangle one can have boring surface spectra (different surfaces look similar) and interesting geography. \memoYF{Need to reconsider this phrasing. }
Therefore, predicting ${\bf \fast }$ and ${\bf s}$ from ${\bf d}$ is degenerate.  

A formally equivalent degeneracy is found in Equation (\ref{eq:d_f_s}) coupled with the conditions (\ref{eq:f_range})-(\ref{eq:s_range}). 
Essentially, the term $\sum _k f_{lk} s_{kj}$ represents the average albedo spectra of $l$-th slice and $W_{il}$ is the matrix that convolve them into the light curves. 
Suppose idealistically that we can retrieve the average spectra of longitudinal slices from the light curves through $W_{il}$; the black trajectory in the bottom panel of Figure \ref{fig:trajectory} represents the color variation along the longitude.  
Again, we can have different sets of $\{ {\bf s}_k \}$ that enclose the averaged albedo spectra of longitudinal slices {\it and} are located in the range of condition (\ref{eq:s_range}), any of which can make up the given averaged albedo spectra with the associated $f_{lk}$. 
Nevertheless, the excursions of the longitudinal colors are more dramatic than the disk-integrated colors of the light curves, thus the possible solutions are somewhat more restricted. 
% For disk-integrated light curves, the color excursions are more muted than the intrinsic longitudinal map, so the degeneracy becomes severer. 


%%%%%%%%%%%%%%%%%%%%%%%%%%%%%%%%%%%%%%%%%%%%%%%%%%%%%%%%%%%%%%
\subsection{``Best Guess'' of the Surface Types}
\label{ss:guess}
%%%%%%%%%%%%%%%%%%%%%%%%%%%%%%%%%%%%%%%%%%%%%%%%%%%%%%%%%%%%%%


%%%%%%%%%%%%%%%%%%%%%%%%%%%%%%%%%%%
\begin{figure*}[tbh!]
   \begin{minipage}{0.33\hsize}
    \begin{center}
\includegraphics[width=\hsize]{mockdata_90deg_3types_t12_lc_noreg.pdf}
    \end{center}
     \end{minipage}   
    \begin{minipage}{0.33\hsize}
    \begin{center}
%\includegraphics[width=\hsize]{mockdata_90deg_3types_t12_lc_noreg.pdf}
\includegraphics[width=\hsize]{mockdata_135deg_3types_t360_lc_noreg.pdf}
    \end{center}
     \end{minipage}
   \begin{minipage}{0.33\hsize}
    \begin{center}
\includegraphics[width=\hsize]{IGBP_lon_noreg.pdf}
    \end{center}
     \end{minipage}
%   \begin{minipage}{0.33\hsize}
%    \begin{center}
%\includegraphics[width=\hsize]{IGBP_PCplane_Nside0.pdf}
%    \end{center}
%     \end{minipage}   
%    \begin{minipage}{0.33\hsize}
%    \begin{center}
%\includegraphics[width=\hsize]{IGBP_PCplane_Nside1.pdf}
%    \end{center}
%     \end{minipage} 
%   \begin{minipage}{0.33\hsize}
%    \begin{center}
%\includegraphics[width=\hsize]{IGBP_PCplane_Nside2.pdf}
%    \end{center}
%     \end{minipage}
%   \begin{minipage}{0.33\hsize}
%    \begin{center}
%\includegraphics[width=\hsize]{mockdata_90deg_3types_t12_lc_reg_l30deg.pdf}
%    \end{center}
%     \end{minipage}      
%   \begin{minipage}{0.33\hsize}
%    \begin{center}
%\includegraphics[width=\hsize]{mockdata_90deg_3types_t12_lc_reg_l40deg.pdf}
%    \end{center}
%     \end{minipage}
    \caption{Upper panels: Probability of surface albedo spectra estimated from the light curves of $\alpha = 90^{\circ }$ (left), from the light curves of $\alpha = 135^{\circ }$ (middle) and from the colors of longitudinal slices. (Lower panels) Making the map coarser using HealPix. Left: 192 pixels, Middle: 48 pixels, Right: 12 pixels. }
\label{fig:PCplane}
\end{figure*}
%%%%%%%%%%%%%%%%%%%%%%%%%%%%%%%%%%%


Given that any vertices in the PC plane that enclose the trajectory/-ies can be a solution, the choice of the solution critically depends on the prior probability distribution. 
% Thus, we need to be careful about the implicit assumptions that are made  
With no assumptions or information on the spectral albedo of the surface types or geography, it may be reasonable to assume that any points in the PC plane except for the inhibited regions are equally likely to correspond to a surface type. 
%While such an assumption does not seem to work at all to choose the surface types, it does put some constrains. 
While such an assumption does not uniquely specify the colors of all $K$ surface types, it does offer constraints due to the relative light curve trajectory within the permitted region. 
%This is because of the relative configuration of the trajectory and the permitted region. 
For example, in Figure \ref{fig:trajectory}, in order to enclose the trajectory with three surface types within the permitted region, we must have at least one point near the bottom left corner of the permitted region (white triangle); this is consistent with the fact that one of the input albedo spectra (ocean) resides there. 

In order to see this more quantitatively, we make a grid on the PC plane  with an interval of 0.2 and consider all combinations of three grid points. 
Then, we assume that all of the combinations that enclose the trajectory are equally likely to be a solution, and find the marginalized probability of having a surface type at each location of the PC plane. 
Specifically, we consider the following Bayesian expression and attempt to find the posterior distribution:
%%%
\begin{equation}
P_{\rm posterior} ( t_{kn} | U_{in} ) = P ( U_{in} | t_{kn} ) \cdot P_{\rm prior} (  t_{kn} ) 
\end{equation}
%%%
where $t_{kn}$ represents the coordinates of the $k$-th surface type on the PC plane ($n$ is either 1 or 2, corresponding to PC 1 and PC 2), i.e., 
%%%
\begin{equation}
s_{kj} = \sum_n t_{kn} V_{nj} + \bar d_j . 
\end{equation}
Assuming that any facet has the same prior probability per area of being a surface type, the prior for each $k$ is:
%%%
\begin{eqnarray}
&& P_{\rm prior} (t_{k1}, t_{k2} ) \, dt_{k1} dt_{k2} \\
&& = \left\{
\begin{array}{ll}
\displaystyle \frac{dt_{k1}  dt_{k2}}{\iint _{\rm allowed} d^2{\bf t}_k} & \;\;\;\;\;\; \mbox{if satisfies (\ref{eq:tilde_s_range})} \\
0 & \;\;\;\;\;\; \mbox{otherwise}
\end{array}
\right.
\end{eqnarray}
%%%
On the other hand, the likelihood is 
%%%
\begin{eqnarray}
&& P ( U_{in} | t_{kn} ) \\
&& \propto \left\{
\begin{array}{ll}
1 \;\;\; & \;\;\;\;\;\; \mbox{if satisfies (\ref{eq:tilde_f_range}) and (\ref{eq:tilde_f_range})} \\
0 \;\;\; & \;\;\;\;\;\; \mbox{otherwise}
\end{array}
\right.
\end{eqnarray}
%%%
Although the graphical meaning of conditions (\ref{eq:tilde_f_range}) and (\ref{eq:tilde_f_sum}) is simply to enclose the data, the judgement is numerically made as follows. 
We first find $\fast $ that satisfies condition (\ref{eq:tilde_f_sum}) from
%%%
\begin{equation}
\begin{pmatrix}
U_{11} & U_{12} & 1 \\
... & & 1 \\
U_{I1} & U_{I2} & 1 
\end{pmatrix}
= 
\begin{pmatrix}
\fast_{11} & \fast_{12} & \fast_{13}  \\
... & \\
\fast_{I1} & \fast_{I2} & \fast_{I3}
\end{pmatrix}
\begin{pmatrix}
t_{11} & t_{12} & 1 \\
t_{21} & t_{22} & 1 \\
t_{31} & t_{32} & 1 
\end{pmatrix}
\end{equation}
%%%
or,
\begin{equation}
\begin{pmatrix}
\fast_{11} & \fast_{12} & \fast_{13}  \\
... & \\
\fast_{I1} & \fast_{I2} & \fast_{I3}
\end{pmatrix}
=
\begin{pmatrix}
U_{11} & U_{12} & 1 \\
... & & 1 \\
U_{I1} & U_{I2} & 1 
\end{pmatrix}
\begin{pmatrix}
t_{11} & t_{12} & 1 \\
t_{21} & t_{22} & 1 \\
t_{31} & t_{32} & 1 
\end{pmatrix}^{-1}
\label{eq:f=ds-1}
\end{equation}
%%%
Note that the last matrix is a $K\times K$ matrix, and thus an inverse exists if $\{ {\bf t}_k \}$ forms a triangle (rather than a straight line or a single dot, in which case $\{ {\bf t}_k \}$ is not a solution anyways).
Then, we ask if the resultant $\fast_{ik}$ are within the range of (\ref{eq:tilde_f_range}). 

Finally, we marginalize the posterior distribution, i.e.,
%%%
\begin{equation} 
P_{\rm posterior} ( {\bf t}_1) = \iint P_{\rm posterior} ( t_{kn} | d_{ij} ) d^2 {\bf t}_2 d^2 {\bf t}_3
\end{equation}
%%%
given the symmetry of ${\bf t}_1$, ${\bf t}_2$, and ${\bf t}_3$. 
$P_{\rm posterior} ( {\bf t}_1) $ is normalized so that it sums up to unity. 

The resultant probability distribution is shown in Figure \ref{fig:PCplane}. 
As expected, we find a relatively definite peak near the point of ocean. 
On the other hand, the peak that corresponds to soil is a blur, and it is almost unconstrained. 
The vegetation in the bottom right corner is moderately constrained. 


%%%%%%%%%%%%%%%%%%%%%%%%%%%%%%%%%%%%%%%%%%%%%%%%%%%%%%%%%%%%%%
\subsection{Conditions for Successful Guess of Surface Spectra}
\label{ss:guess}
%%%%%%%%%%%%%%%%%%%%%%%%%%%%%%%%%%%%%%%%%%%%%%%%%%%%%%%%%%%%%%


We reiterate that in this framework the success of the surface retrieval critically depends on the relative configuration of the trajectory and the allowed region. 
More specifically, the good constraints on the colors of ocean appears to be due to the combination of the large covering fraction of ocean and the location of the color of ocean close to a corner in the PC plane (due to low albedo at longer wavelength). 

In order to illustrate these two effects and see under what conditions we can faithfully retrieve surface spectra, we created two additional mock light curves, (a) with the geographical maps of the soil and vegetation being swapped, and  (b) with the geographical maps of the soil and ocean being swapped. 
The left and right panels of Figure \ref{fig:swap} correspond to (a) and (b), respectively. 

%%%%%%%%%%%%%%%%%%%%%%%%%%%%%%%%%%%
\begin{figure}[htb!]
    \begin{center}
    \includegraphics[width=\hsize]{swap.pdf}
    \end{center}
    \caption{Left: Trajectory of the light curves with the geographical maps of the soil and vegetation being swapped, and the resultant color contour of the posterior probability of albedo spectra of surface types. (b) Same as (a) but with the geographical maps of the soil and ocean being swapped.}
\label{fig:swap}
\end{figure}
%%%%%%%%%%%%%%%%%%%%%%%%%%%%%%%%%%%

When we swap soil and vegetation (left panel) the covering fraction of vegetation becomes smaller, and the trajectory of the light curves does not approach the vegetation endmember. As a result, the constraints of the color of vegetation becomes weaker. Thus, a large covering fraction of the surface type tend to yield a better constraint on the albedo spectra of that type. 

Meanwhile, in (b) where we swapped the soil and ocean, although soil is the dominant component, the true location of the soil is rather not preferred based on our analysis to other locations toward the top, because there are larger space there. 
Therefore, large covering fraction is not a sufficient condition to better constrain the spectral albedo. 
In other words, a surface spectrum can be well retrieved if it has favorable geography (ideally an entire hemisphere solely covered with that surface) and it has favorable albedo spectrum (albedo close to 0 or 1 at as many wavelengths as possible). 



%%%%%%%%%%%%%%%%%%%%%%%%%%%%%%%%%%%%%%%%%%%%%%%%%%%%%%%%%%%%%%
\subsection{Additional Assumptions}
\label{ss:regularization}
%%%%%%%%%%%%%%%%%%%%%%%%%%%%%%%%%%%%%%%%%%%%%%%%%%%%%%%%%%%%%%

Occasionally, we may want to adopt additional constraints such as the smoothness of the covering fraction as a function of time/longitude, and/or the smoothness of the albedo spectra as a function of wavelength. 
Their smoothness can be imposed as regularization terms with appropriate correlation lengths. 

The regularization for the smoothness of albedo spectra as a function of wavelength may be regarded as a prior probability distribution on the PC plane. 
For example, in Figure \ref{fig:PCplane}, albedo spectra corresponding to the upper part of the permitted region exhibits a dramatic decrease in albedo from 0.65$\mu $m to $0.75\mu $m, which we may think is less likely to be the color of a physical surface just because we do not think of a surface type with such a feature. 
It should always be noted, however, that such constraints are based upon our prior assumptions and may not be true. 
For example, albedo spectra of vegetation have the sharp edge known as red edge. 
In addition, in a realistic case with an atmosphere, absorption by atmospheric molecules can also produce sharp changes in albedo spectra. 




%%%%%%%%%%%%%%%%%%%%%%%%%%%%%%%%%%%%%%%%%%%%%%%%%%%%%%%%%%%%%%
\section{Application to EPOXI data}
\label{s:EPOXI}
%%%%%%%%%%%%%%%%%%%%%%%%%%%%%%%%%%%%%%%%%%%%%%%%%%%%%%%%%%%%%%

%%%%%%%%%%%%%%%%%%%%%%%%%%%%%%%%%%%
\begin{figure}[tbh!]
    \begin{center}
	\includegraphics[width=\hsize]{raddata_2_PCplane_noreg_labels.pdf}
%	\includegraphics[width=\hsize]{raddata_2_norm_Mj_PCs.pdf}
	\includegraphics[width=\hsize]{raddata_2_norm_spectra.pdf}
    \end{center}
    \caption{Upper panel: PC plane from EPOXI observation of June 2008. \memoYF{I have not computed the exact location of ocean, soil, or clouds, but roughly, ocean is probably around (-0.2, 0.4), soil is around (0.5, 0.0), and cloud is around (0.1,-0.5).} \memoNBC{This plot doesn't use regularization, correct?}\memoYF{Correct, it does not use regularization.} Lower panel: albedo spectra of the upper left peak in the upper panel. }
\label{fig:EPOXI}
\end{figure}
%%%%%%%%%%%%%%%%%%%%%%%%%%%%%%%%%%%

In this section, we apply our procedure to the observed multi-band light curves of the Earth observed by EPOXI mission. 
Specifically we use the same datasets as \citet{Cowan2013}, i.e., the 7-band diurnal light curves observed in June of 2008 \citep{Livengood2011}. 
Using two dominant PCs we extracted, we performed the same analysis described in the previous section 

Figure \ref{fig:EPOXI} shows the trajectory of the light curve projected onto the PC plane, with the grayed region being the forbidden region. 
While the permitted region appears to be close to a square, it is in fact a heptagon, bounded by 7 inequalities (2D slice through a 7D hypercube). 
Overall, the upper boundaries come from the requirement of $s_{kj}>0$ while the lower boundaries comes from the requirement of $s_{kj}<1$. 

In this case, the trajectory of planetary light curves is far from the boundary because the color variations of the real Earth are muted by the cloud cover. 
Consequently, we do not see three peaks but a spectrum of the potential. \memoNBC{Likelihood?} \memoYF{Posterior probability would be the better word, probably}. 
Nevertheless, we are likely to find one surface type near the upper right part of the allowed region. 




%%%%%%%%%%%%%%%%%%%%%%%%%%%%%%%%%%%%%%%%%%%%%%%%%%%%%%%%%%%%%%
\section{Discussion}
\label{s:discussion}
%%%%%%%%%%%%%%%%%%%%%%%%%%%%%%%%%%%%%%%%%%%%%%%%%%%%%%%%%%%%%%

\memoYF{This section could be merged with Conclusion section.}

\subsection{Effect of Uncertain Planetary Radius and Orbit}

So far we have assumed that the geometrical parameters are completely known, including the planetary orbit, planetary spin axis and the planetary radius. 
% In this section we discuss the influence of incomplete knowledge of these parameters. 
How would the uncertainties in these parameters affect our retrieval?
In fact, the decomposition into $\fast $ and $s$ does not require the information of spin axis or spin period. 
What matters most in this case is the uncertainty in the normalization of apparent albedo. 
In reality, direct imaging observations alone provide the intensity of the 
planetary light and the intensity of the stellar light, and in order to convert this to the unit of albedo we need to know the orbital distance from the star, phase angle, and planetary radius. 
If there are uncertainties in these estimates, the relative configuration among the light curve trajectories, the boundaries coming from $s_{jk} < 1$, and those coming from $s_{jk} > 0$ are changed \memoNBC{Why?}, thus the estimates can be substantially biased. 

\subsection{Deviation from Lambert's Law}
\label{ss:deviateLambert}

We have assumed that the surface scattering obeys Lambert's law, but in reality they are not perfect Lambert scatterers, as noted. 
In particular, scattering by ocean exhibits prominent specular reflection and the reflectivity increases when the incident light is grazing, or equivalently at crescent phase \citep[e.g.,][]{Williams2008}. 
Furthermore, when we take account of an atmosphere above the surface, the albedo spectra of surface overlaid by an atmosphere changes with phase angle.  
Considering these effects, the trajectories of the light curves at different phases do not have to reside in the same PC plane. 
These light curves can be analyzed independently, and the components varying with phase would give us insights into the anisotropic scatterers. 


%%%%%%%%%%%%%%%%%%%%%%%%%%%%%%%%%%%%%%%%%%%%%%%%%%%%%%%%%%%%%%
\section{Conclusion}
\label{s:conclusion}
%%%%%%%%%%%%%%%%%%%%%%%%%%%%%%%%%%%%%%%%%%%%%%%%%%%%%%%%%%%%%%

In this paper, we revisited the framework to estimate the albedo spectra of major surface types of exoplanets and their distribution from disk-integrated multi-band photometric data. 
We pointed out the inherit degeneracy which makes it impossible to find unique solutions. 
While admitting the degeneracy, some constraints about the likely solutions may be found, owing to the nature of albedo which is between 0 and 1. 
We demonstrate using a simplified toy model of the Earth that such constraints work in some cases. 
...
%The constraints can be...


\acknowledgements

We acknowledge Exo-Cartography workshop held in 2016 under the support of International Space Science Institute (ISSI). 
Y. F. acknowledges the generous support from Universities Space Research Association through an appointment to the NASA Postdoctoral Program at the NASA Goddard Institute for Space Studies. 
Y. F. also acknowledges support from the NASA Astrobiology Program through the Nexus for Exoplanet System Science.
McGill Space Institute, Institut de recherche sur les exoplan\'etes? 


\bibliography{ref}

\appendix




%%%%%%%%%%%%%%%%%%%%%%%%%%%%%%%%%%%%%%%%%%%%%%%%%%%%%%%%%%%%%%
\section{Difficulty in Longitudinal Observations}
%%%%%%%%%%%%%%%%%%%%%%%%%%%%%%%%%%%%%%%%%%%%%%%%%%%%%%%%%%%%%%

\memoYF{Looking for better description. }

The difficulty in color retrieval from the rotational light curves is simply that we cannot resolve the planetary surface with high spatial resolution; if we had extremely high spatial resolution of the planetary surface, the colors of individual surface patches would be well represented by the colors of either ocean, soil, or vegetation, and it would be trivial to identify the surface types. 
Figure \ref{fig:lowresolution} shows how lowering the spatial resolution of the map mixes the colors of individual pixels, using HEALPix pixelization. 
From left to right, we change the resolution from 192 pixels, 48 pixels to 12 pixels, starting from the map in Figure \ref{fig:mockdata}. 
As expected, the fraction of pixels with intermediate colors increases as we lower the resolution. 
When the number of pixel is as small as 12, there is no longer a pixel composed purely of soil or vegetation. 
With 48 pixels, the triangular shape of the locus of points is marginally seen. 

In the case of observations of diurnal light curves, as we go closer to crescent phase, the illuminated and visible area becomes narrower and the colors go more extreme (Figure \ref{fig:trajectory}), which is better for   constraining the colors of the surface types. 
However, changing the width of the illuminated and visible area is different from changing the spatial resolution using HEALPix as described above. 
This is because even at crescent phase, colors of very distant pixels, beyond the correlation length of the geography, are mixed together---for example, the colors of the arctic, tropics, and antarctic are mixed in an equatorial observation, no matter how extreme the orbital phase is. 
% On the other hand, lowering resolution in HEALPix maintain the 
Figure \ref{fig:lowresolution} should thus be compared with the trajectory in the upper right panel of Figure \ref{fig:PCplane}, which is the colors of 360 longitudinal slices (width: $1^{\circ }$). 
Despite the number of slices, the colors of individual slices never approach the colors of vegetation or soil. 
This limits our ability to retrieve the surface types. 


%%%%%%%%%%%%%%%%%%%%%%%%%%%%%%%%%%%
\begin{figure*}[tbh!]
   \begin{minipage}{0.33\hsize}
    \begin{center}
\includegraphics[width=\hsize]{IGBP_PCplane_Nside2.pdf}
    \end{center}
     \end{minipage}   
    \begin{minipage}{0.33\hsize}
    \begin{center}
\includegraphics[width=\hsize]{IGBP_PCplane_Nside1.pdf}
    \end{center}
     \end{minipage}
   \begin{minipage}{0.33\hsize}
    \begin{center}
\includegraphics[width=\hsize]{IGBP_PCplane_Nside0.pdf}
    \end{center}
     \end{minipage}
    \caption{Making the map coarser using HealPix. Left: 192 pixels, Middle: 48 pixels, Right: 12 pixels. }
\label{fig:lowresolution}
\end{figure*}
%%%%%%%%%%%%%%%%%%%%%%%%%%%%%%%%%%%



\end{document}



%%%%%%%%%%%%%%%%%%%%%%%%%%%%%%%%%%%%%%%%%%%%%%%%%%%%%%%%%%%%%%
\section{Swapping Geographical Maps}
%%%%%%%%%%%%%%%%%%%%%%%%%%%%%%%%%%%%%%%%%%%%%%%%%%%%%%%%%%%%%%

% In this section we demonstrate how the albedo spectra and geographical maps of the surface types affect our retrieval. 
In our fiducial model, the color of ocean was best constrained (see Figure \ref{fig:PCplane}). 
%
This appears to be due to the combination of the large covering fraction of ocean and the location of the color of ocean close to a corner in the PC plane (due to low albedo at longer wavelength). 
In other words, something about the geography and/or color of ocean makes it relatively easy to retrieve. 
To illustrate these two effects, we created two additional mock light curves, (a) with the geographical maps of the soil and vegetation being swapped, and  (b) with the geographical maps of the soil and ocean being swapped. 
The left and right panels of Figure \ref{fig:swap} correspond to (a) and (b), respectively. 
%
When we swap soil and vegetation (left panel) the covering fraction of vegetation becomes smaller, and the trajectory of the light curves does not approach the vegetation endmember. As a result, the constraints of the color of vegetation becomes weaker. Thus, a large covering fraction of the surface type tend to yield a better constraint on the albedo spectra of that type. 
%
Meanwhile, in (b) where we swapped the soil and ocean, although soil is the dominant component, the true location of the soil is rather not preferred based on our analysis to other locations toward the top, because there are larger space there. 
Therefore, large covering fraction is not a sufficient condition to better constrain the spectral albedo. 
In other words, a surface spectrum can be well retrieved if it has favorable geography (ideally an entire hemisphere solely covered with that surface) and it has favorable albedo spectrum (albedo close to 0 or 1 at as many wavelengths as possible). 

%%%%%%%%%%%%%%%%%%%%%%%%%%%%%%%%%%%
\begin{figure}[tbh!]
    \begin{center}
   \begin{minipage}{0.33\hsize}
\includegraphics[width=\hsize]{mockdata_90deg_3types2_t360_lc_noreg.pdf}
     \end{minipage}
   \begin{minipage}{0.33\hsize}
%    \begin{center}
\includegraphics[width=\hsize]{mockdata_90deg_3types3_t360_lc_noreg.pdf}
%    \end{center}
     \end{minipage}
    \end{center}
    \caption{Left: Trajectory of the light curves with the geographical maps of the soil and vegetation being swapped, and the resultant color contour of the posterior probability of albedo spectra of surface types. (b) Same as (a) but with the geographical maps of the soil and ocean being swapped. \memoNBC{Show the corresponding maps!} }
\label{fig:swap}
\end{figure}
%%%%%%%%%%%%%%%%%%%%%%%%%%%%%%%%%%%
